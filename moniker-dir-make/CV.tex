%%%%%%%%%%%%%%%%%%%%%%%%%%%%%%%%%%%%%%%%%%%%%%%%%%%%%%%%%%%%%%%%%%%%%%%%
%%%%%%%%%%%%%%%%%%%%%% Simple LaTeX CV Template %%%%%%%%%%%%%%%%%%%%%%%%

%% NOTE: If you find that it says                                     %%
%%                                                                    %%
%%                           1 of ??                                  %%
%%                                                                    %%
%% at the bottom of your first page, this means that the AUX file     %%
%% was not available when you ran LaTeX on this source. Simply RERUN  %%
%% LaTeX to get the ``??'' replaced with the number of the last page  %%
%% of the document. The AUX file will be generated on the first run   %%
%% of LaTeX and used on the second run to fill in all of the          %%
%% references.                                                        %%
%%%%%%%%%%%%%%%%%%%%%%%%%%%%%%%%%%%%%%%

\RequirePackage{mymoniker}    % Replace CarrollCD in mymoniker.sty with your LastnameFirstNameMiddleName
\RequirePackage{dissertation} % required extra stuff

% Don't like 10pt? Try 11pt or 12pt
\documentclass[12pt]{article}

% This is a helpful package that puts math inside length specifications
\usepackage{calc}
\usepackage{pifont}
\usepackage{marvosym}


% Simpler bibsection for CV sections
% (thanks to natbib for inspiration)
\makeatletter
\newlength{\bibhang}
\setlength{\bibhang}{1em}
\newlength{\bibsep}
 {\@listi \global\bibsep\itemsep \global\advance\bibsep by\parsep}
\newenvironment{bibsection}%
        {\vspace{-\baselineskip}\begin{list}{}{%
       \setlength{\leftmargin}{\bibhang}%
       \setlength{\itemindent}{-\leftmargin}%
       \setlength{\itemsep}{\bibsep}%
       \setlength{\parsep}{\z@}%
        \setlength{\partopsep}{0pt}%
        \setlength{\topsep}{0pt}}}
        {\end{list}\vspace{-.6\baselineskip}}
\makeatother

% Layout: Puts the section titles on left side of page
\reversemarginpar

%
%         PAPER SIZE, PAGE NUMBER, AND DOCUMENT LAYOUT NOTES:
%
% The next \usepackage line changes the layout for CV style section
% headings as marginal notes. It also sets up the paper size as either
% letter or A4. By default, letter was used. If A4 paper is desired,
% comment out the letterpaper lines and uncomment the a4paper lines.
%
% As you can see, the margin widths and section title widths can be
% easily adjusted.
%
% ALSO: Notice that the includefoot option can be commented OUT in order
% to put the PAGE NUMBER *IN* the bottom margin. This will make the
% effective text area larger.
%
% IF YOU WISH TO REMOVE THE ``of LASTPAGE'' next to each page number,
% see the note about the +LP and -LP lines below. Comment out the +LP
% and uncomment the -LP.
%
% IF YOU WISH TO REMOVE PAGE NUMBERS, be sure that the includefoot line
% is uncommented and ALSO uncomment the \pagestyle{empty} a few lines
% below.
%

%% Use these lines for letter-sized paper
%\usepackage[paper=letterpaper,
%           %includefoot, % Uncomment to put page number above margin
%            marginparwidth=0.7in,     % Length of section titles
%            marginparsep=.05in,       % Space between titles and text
%            margin=0.5in,               % 1 inch margins
%            includemp]{geometry}

% Use these lines for A4-sized paper
\usepackage[paper=a4paper,
            %includefoot, % Uncomment to put page number above margin
            marginparwidth=24mm,    % Length of section titles
            marginparsep=1mm,       % Space between titles and text
            margin=15mm,              % 25mm margins
            includemp]{geometry}

%% More layout: Get rid of indenting throughout entire document
\setlength{\parindent}{0in}

%% This gives us fun enumeration environments. compactitem will be nice.
\usepackage{paralist}
\usepackage[shortlabels]{enumitem}
% \usepackage[misc]{ifsym}
%% Reference the last page in the page number
%
% NOTE: comment the +LP line and uncomment the -LP line to have page
%       numbers without the ``of ##'' last page reference)
%
% NOTE: uncomment the \pagestyle{empty} line to get rid of all page
%       numbers (make sure includefoot is commented out above)
%
\usepackage{fancyhdr,lastpage}
\pagestyle{fancy}
%\pagestyle{empty}      % Uncomment this to get rid of page numbers
\fancyhf{}\renewcommand{\headrulewidth}{0pt}
\fancyfootoffset{\marginparsep+\marginparwidth}
\newlength{\footpageshift}
\setlength{\footpageshift}
          {0.1\textwidth+0.1\marginparsep+0.1\marginparwidth-2in}
\lfoot{\hspace{\footpageshift}%
       \parbox{3.5in}{\, \hfill %
                    \arabic{page} of \protect\pageref*{LastPage} % +LP
%                    \arabic{page}                               % -LP
                    \hfill \,}}

% Finally, give us PDF bookmarks
\usepackage{color,hyperref}
\definecolor{darkblue}{rgb}{0.0,0.0,0.3}
\hypersetup{colorlinks,breaklinks,
            linkcolor=darkblue,urlcolor=darkblue,
            anchorcolor=darkblue,citecolor=darkblue}


\usepackage{ifthen}
\newboolean{long}
\setboolean{long}{false}
\newboolean{industry}
\setboolean{industry}{false}

%%%%%%%%%%%%%%%%%%%%%%%% End Document Setup %%%%%%%%%%%%%%%%%%%%%%%%%%%%


%%%%%%%%%%%%%%%%%%%%%%%%%%% Helper Commands %%%%%%%%%%%%%%%%%%%%%%%%%%%%

% The title (name) with a horizontal rule under it
%
% Usage: \makeheading{name}
%
% Place at top of document. It should be the first thing.
\newcommand{\makeheading}[1]%
        {\hspace*{-\marginparsep minus \marginparwidth}%
         \begin{minipage}[t]{\textwidth+\marginparwidth+\marginparsep}%
                {\large \bfseries #1}\\[-0.15\baselineskip]%
                 \rule{\columnwidth}{1pt}%
         \end{minipage}}

% The section headings
%
% Usage: \section{section name}
%
% Follow this section IMMEDIATELY with the first line of the section
% text. Do not put whitespace in between. That is, do this:
%
%       \section{My Information}
%       Here is my information.
%
% and NOT this:
%
%       \section{My Information}
%
%       Here is my information.
%
% Otherwise the top of the section header will not line up with the top
% of the section. Of course, using a single comment character (%) on
% empty lines allows for the function of the first example with the
% readability of the second example.
\renewcommand{\section}[2]%
        {\pagebreak[2]\vspace{1\baselineskip}%
         \phantomsection\addcontentsline{toc}{section}{#1}%
         \hspace{0in}%
         \marginpar{
         \raggedright \scshape #1}#2}

% An itemize-style list with lots of space between items
\newenvironment{outerlist}[1][\enskip\textbullet]%
        {\begin{itemize}[#1]}{\end{itemize}%
         \vspace{-0.6\baselineskip}}

% An environment IDENTICAL to outerlist that has better pre-list spacing
% when used as the first thing in a \section
\newenvironment{lonelist}[1][\enskip\textbullet]%
        {\vspace{-\baselineskip}\begin{list}{#1}{%
        \setlength{\partopsep}{0pt}%
        \setlength{\topsep}{0pt}}}
        {\end{list}\vspace{-.6\baselineskip}}

% An itemize-style list with little space between items
% \newenvironment{innerlist}[1][\enskip\textbullet]%
\newenvironment{innerlist}[1][\enskip$\circ$]%
        {\begin{compactitem}[#1]}{\end{compactitem}}

% An environment IDENTICAL to innerlist that has better pre-list spacing
% when used as the first thing in a \section
\newenvironment{loneinnerlist}[1][\enskip\textbullet]%
        {\vspace{-\baselineskip}\begin{compactitem}[#1]}
        {\end{compactitem}\vspace{-.6\baselineskip}}

% To add some paragraph space between lines.
% This also tells LaTeX to preferably break a page on one of these gaps
% if there is a needed pagebreak nearby.
\newcommand{\blankline}{\quad\pagebreak[2]}

% Uses hyperref to link DOI
\newcommand\doilink[1]{\href{http://dx.doi.org/#1}{#1}}
\newcommand\doi[1]{doi:\doilink{#1}}


%%%%%%%%%%%%%%%%%%%%%%%% End Helper Commands %%%%%%%%%%%%%%%%%%%%%%%%%%%

%%%%%%%%%%%%%%%%%%%%%%%%% Begin CV Document %%%%%%%%%%%%%%%%%%%%%%%%%%%%

%\hyphenpenalty = 9999
\def\vs{\vspace{-0.1in}}
\begin{document}
% \makeheading{Curriculum Vitae\\ [0.3cm] TIEP HUU VU\quad~~~~~~\quad\quad\quad\quad\quad\quad\quad\quad\quad\quad\quad\quad\quad\quad{\small Last update: December 17, 2015}}
\makeheading{Tao Wang \hfill {\small Last update: \today}}


\section{Contact Information}

\newlength{\rcollength}\setlength{\rcollength}{3 in}
\vs
\begin{tabular}[t]{@{}p{\textwidth-\rcollength}p{\rcollength}}
Department of Economics  & \texttt{Homepage:}\href{http://taowangecon.github.io}{http://taowangecon.github.io}\\
Johns Hopkins University & \texttt{GitHub:} \href{https://github.com/iworld1991}{https://github.com/iworld1991}\\
Baltimore, MD 21211 &  {\large\Letter} \texttt{E-mail:}\href{mailto:twang80@jhu.edu}{twang80@jhu.edu} \\
Work: (410) 516-7601  & Home: (443) 467-4300
\end{tabular} 
%%

%==============================================================
\vspace{0.2in}
\section{Fields of Research} % (fold)
\label{sec:research_backg}
\vspace{-0.25in}
\begin{outerlist}
	\item {\bf INTEREST 1}: MORE DETAILS	\item {\bf INTEREST 2}: MORE DETAILS

\end{outerlist}
% section research_backg (end)
%% =========  ==============================


\section{Education}
\href{https://www.jhu.edu}{\textbf{Johns Hopkins University}}, Baltimore, MD \hfill 2017-- 2023 (expected)
\begin{outerlist}
	\item M.A. and Ph.D.  in \href{https://econ.jhu.edu} {Economics}
	 \item Thesis: \textit{THESIS TITLE}
	\ifthenelse{\boolean{industry}}{}  {\item Principal Advisor: Prof. \href{https://econ.jhu.edu/directory/christopher-carroll/}{Christopher Carroll} 
		%\item Coursework: Time Series Econometrics, Asset Pricing, Decision Making under Uncertainty, Computational Macroeconomics, Information in Economics and Finance, International Finance, Advanced Macroeconomics I \& II
	}
\end{outerlist}


\vspace{0.1in}
\href{https://www.xxxx.com}{\textbf{XXX University}}, LOCATION, \hfill 2009--2013
\begin{outerlist}
	\item B.A. in Economics 
	\ifthenelse{\boolean{long}}{   \item Thesis: \textit{THESIS TITLE}}{}
\end{outerlist}



%% ================== block:  ==========================
\section{Working Papers}
\vspace{-.25in}
\begin{enumerate}
	\item  \textit{``Perceived versus Calibrated Income Risks in Heterogeneous-Agent Consumption Models''}(Job Market Paper), \href{https://github.com/iworld1991/PIR/blob/master/PIR.pdf}{working draft}, 2022.
	\ifthenelse{\boolean{long}}{
		\begin{innerlist}
			\item[] \textit{Abstract}:  Models of microeconomic consumption (including those used in HA-macro models) typically calibrate the size of income risk to match panel data on household income dynamics. But, for several reasons, what is measured as risk from such data may not correspond to the risk perceived by the agent. This paper instead uses data from the New York Fed's \textit{Survey of Consumer Expectations} to directly calibrate perceived income risks. One of several examples of the implications of heterogeneity in perceived income risks is increased wealth inequality stemming from differential precautionary saving motivations. I also explore the implications of the fact that the perceived risk is lower than the calibrated level.
		\end{innerlist}
	}{}
	
	\item \textit{``How Do Agents Form Inflation Expectations? Evidence from Cross-moment Estimation and the Uncertainty''}, \href{https://taowangecon.github.io/papers/InfVar.pdf}{working paper}, 2021.
	\ifthenelse{\boolean{long}}{
		\begin{innerlist}
			
			\item[] \textit{Abstract}: Density forecasts of macroeconomic variables provide one additional moment restriction, uncertainty, for testing and exploring the implications of theories about how people form expectations differently from full-information rationality benchmark. This paper first documents the persistent dispersion in inflation uncertainty of professionals and households, and how it conveys different in- formation from the widely used proxies to uncertainty such as cross-sectional disagreement and forecast errors. Second, utilizing the panel data structure of both surveys, I provide additional reduced-form test results as well as structural estimates for three workhorse theories of “irrational expectation”: sticky expectations (SE), noisy information (NI), and diagnostic expectations (DE) by jointly accounting for its predictions for different moments. This is a natural extension of Coibion and Gorodnichenko (2012), which examines different moments separately. Also, motivated by the time-varying pattern of the uncertainty observed from surveys, I considers an alternative inflation process featuring stochastic volatility. These extensions allow me to match the joint dynamics of inflation and forecast moments in better goodness of fit. It also testifies how incorporating higher moments from survey data helps understand both the expectation formation mechanisms and inflation dynamics.
		\end{innerlist}
	}{}
	\item  \textit{``Learning from Friends in a Pandemic: Social Networks and the Macroeconomic Response of Consumption''}, \href{https://taowangecon.github.io/papers/Makridis,Wang\%20-\%20Learning\%20from\%20Friends.pdf}{working paper}, with Christos Makridis, submitted.
	\ifthenelse{\boolean{long}}{
		\begin{innerlist}
			\item[] \textit{Abstract}: We extend a standard incomplete-market macro model to study how social networks affect
			households consumption via macroeconomic expectations. To motivate the model, we use exogenous variation in the exposure of counties to COVID-19 shocks in their social network to show that a 10\% rise in the number of cases and deaths is associated with a 0.15\% and 0.42\% decline in consumption expenditures, respectively. Interestingly, these effects are concentrated	among consumer goods and services that rely more on social contact and are not driven by local shocks. Next, we embed a tractable belief formation mechanism through social communications, a la DeGroot (1974) in an otherwise standard heterogeneous-agent model and calibrate the model to replicate our micro findings. We show a pandemic-augmented version of this model where infections initially hit a fraction of more connected regions and gradually propagated via social network helps generates macroeconomic dynamics more aligned with the empirical patterns of aggregate consumption and cross-sectional heterogeneity. We demonstrate how the dynamic and size of aggregate responses depend on the location of the initial shocks and the structure of the network.
		\end{innerlist}
	}{}
	
	\item  \textit{``Epidemiological Expectations''}, \href{https://github.com/llorracc/EpiExp/blob/master/draft/chapter/Book.pdf}{working paper}, in preparation for \textit{Handbook of Economic Expectations}, with Christopher Carroll, submitted.
	\ifthenelse{\boolean{long}}{
		\begin{innerlist}
			\item[] \textit{Abstract}: `Epidemiological' models of belief formation put social interactions at their core; such models are the main (almost, the only) tool used by non-economists to study the dynamics of beliefs in populations.  We survey the (comparatively) small literature in which economists attempting to model the consequences of beliefs about the future -- `expectations' -- have employed a full-fledged epidemiological approach to explore an economic question.  We draw connections to related work on `contagion,' narrative economics, news/rumor spreading, and the spread of online content. We conclude by arguing that a number of independent developments have converged in ways that may make EE modeling more feasible and more appealing in the past.
		\end{innerlist}
	}{}
	
	\item  \textit{``Perceived (Un)employment Risks Over Business Cycles''}, work in progress, with  William Du and Xincheng Qiu, 2022.
	\ifthenelse{\boolean{long}}{
		\begin{innerlist}
			\item[] \textit{Abstract}: How do income risks such as job loss and finding probabilities as perceived by households change over business cycles? Are the perceptions of income risks reported in the survey consistent with the realizations estimated from the conventional data sources? Answering these questions could help inform both the rapidly rising incomplete-market macro literature that shows the change in income risks is an important driver of business cycle fluctuations and the behavioral extensions of these models. The direct empirical evidence relied upon expectation surveys has been rare, as survey-reported beliefs about income risks are not available until the recent decade. This paper achieves two goals in light of these developments. First, we impute survey-reported beliefs of unemployment risks from the recent decade back to as early as the 1970s, utilizing the cross-survey correlations between the risk beliefs and other macroeconomic expectations available for a longer time span. Second, we document a number of salient facts about the changes in unemployment risk perceptions over business cycles, and also how it compares to realizations estimated from the census data. We found that although the beliefs do reasonably respond to real-time macroeconomic conditions and co-move with their realizations ex-post, households' belief underact to changes of aggregate labor market conditions.
		\end{innerlist}
	}{}
\end{enumerate}   

%%=====================


\section{Conference} % (fold)
\label{sec:conference}
\vspace{-0.25in}
\begin{outerlist}
	\item 	 Midwest Macro, XXX, XXX 
\end{outerlist}




\section{Awards} % (fold)
\label{sec:award}
\vspace{-0.25in}
\begin{outerlist}
	\item 	 Joel Dean Undergraduate Teaching Award, 2019-2020, 2020-2021, 2021-2022, Johns Hopkins University
	\item Sylvia and Wilifried Prewo Fellowship, 2020-2021, Johns Hopkins University 
\end{outerlist}


%% =========  ==============================
\section{Research Assistant Experience} % (fold)
\label{sec:research_exper}
\vspace{-0.25in}
\begin{outerlist}
		\item {\bf NAME of PROJECT 1}  \hfill 2021-present\\
	XXX project headed by Prof.XXX
	\begin{innerlist}
		\item Implemented natural language processing models such as LDA
	
	\end{innerlist}
	\item {\bf NAME of PROJECT 2} \hfill 2019-present\\
	\href{https://econ-ark.org}{Econ-ARK Project} headed by Prof. Christopher Carroll
	\begin{innerlist}
		\item Developed \href{https://github.com/llorracc/SolvingMicroDSOPs/blob/master/Code/Python/SolvingMicroDSOP-Python.ipynb}{Python solution methods of dynamic choice problems}
		
	\end{innerlist}
	
\end{outerlist}


\section{Employment Experiences}
\vspace{-0.25in}
\begin{outerlist}
	\item \textbf{NAME of INSTITUTION 1} \hfill 2016-2017 \\
	Research Assistant XXX
	\item \textbf{NAME of INSTITUTION 2} \hfill 2015-2016 \\
	XXXX
\end{outerlist}



%% =========  ==============================
\section{Technical Skills} % (fold)
\label{sec:technical_ski}
\vspace{-0.25in}
\begin{outerlist}
	\item {\it Programming Languages}: Python, Matlab, Stata, HTML
	\item {\it Languages}: Chinese Mandarin (native), English (proficient), Spanish (conversational)
\end{outerlist}


%% ================== block:  ==========================
\section{Policy Publication}
\vspace{-.25in}
\begin{enumerate}
	\item {\bf XXX and XXX }. \textit{``TITLE of PUBLICATION''}, 2018
\end{enumerate} 
%% ==============================================================


\ifthenelse{\boolean{industry}}{}{
	\section{References}
	\begin{itemize}
		%\item[]\def\halfblankline{\vspace{0.1in}}
		\item Prof. \href{http://www.econ2.jhu.edu/people/ccarroll/index.html}{\textbf{Christopher Carroll}} (JHU), \href{mailto:ccarroll@jhu.edu} {ccarroll@jhu.edu}
		
		\item Prof. \href{https://econ.jhu.edu/directory/jonathan-wright/}{\textbf{Jonathan Wright}} (JHU),  \href{mailto:wrightj@jhu.edu} {wrightj@jhu.edu}
		
		
		\item Prof. \href{https://sites.google.com/view/francescobianchi/home}{\textbf{Francesco Bianchi}} (JHU),  \href{mailto:francesco.bianchi@jhu.edu} {francesco.bianchi@jhu.edu}
	\end{itemize}
}

\end{document}