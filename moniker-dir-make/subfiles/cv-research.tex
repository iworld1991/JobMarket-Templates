
%% ================== block:  ==========================
\section{Working Papers}
\vspace{-.25in}
\begin{enumerate}
	\item  \textit{``Perceived versus Calibrated Income Risks in Heterogeneous-Agent Consumption Models''}(Job Market Paper), \href{https://github.com/iworld1991/PIR/blob/master/PIR.pdf}{working draft}, 2022.
	\ifthenelse{\boolean{long}}{
		\begin{innerlist}
			\item[] \textit{Abstract}:  Models of microeconomic consumption (including those used in HA-macro models) typically calibrate the size of income risk to match panel data on household income dynamics. But, for several reasons, what is measured as risk from such data may not correspond to the risk perceived by the agent. This paper instead uses data from the New York Fed's \textit{Survey of Consumer Expectations} to directly calibrate perceived income risks. One of several examples of the implications of heterogeneity in perceived income risks is increased wealth inequality stemming from differential precautionary saving motivations. I also explore the implications of the fact that the perceived risk is lower than the calibrated level.
		\end{innerlist}
	}{}
	
	\item \textit{``How Do Agents Form Inflation Expectations? Evidence from Cross-moment Estimation and the Uncertainty''}, \href{https://taowangecon.github.io/papers/InfVar.pdf}{working paper}, 2021.
	\ifthenelse{\boolean{long}}{
		\begin{innerlist}
			
			\item[] \textit{Abstract}: Density forecasts of macroeconomic variables provide one additional moment restriction, uncertainty, for testing and exploring the implications of theories about how people form expectations differently from full-information rationality benchmark. This paper first documents the persistent dispersion in inflation uncertainty of professionals and households, and how it conveys different in- formation from the widely used proxies to uncertainty such as cross-sectional disagreement and forecast errors. Second, utilizing the panel data structure of both surveys, I provide additional reduced-form test results as well as structural estimates for three workhorse theories of “irrational expectation”: sticky expectations (SE), noisy information (NI), and diagnostic expectations (DE) by jointly accounting for its predictions for different moments. This is a natural extension of Coibion and Gorodnichenko (2012), which examines different moments separately. Also, motivated by the time-varying pattern of the uncertainty observed from surveys, I considers an alternative inflation process featuring stochastic volatility. These extensions allow me to match the joint dynamics of inflation and forecast moments in better goodness of fit. It also testifies how incorporating higher moments from survey data helps understand both the expectation formation mechanisms and inflation dynamics.
		\end{innerlist}
	}{}
	\item  \textit{``Learning from Friends in a Pandemic: Social Networks and the Macroeconomic Response of Consumption''}, \href{https://taowangecon.github.io/papers/Makridis,Wang\%20-\%20Learning\%20from\%20Friends.pdf}{working paper}, with Christos Makridis, submitted.
	\ifthenelse{\boolean{long}}{
		\begin{innerlist}
			\item[] \textit{Abstract}: We extend a standard incomplete-market macro model to study how social networks affect
			households consumption via macroeconomic expectations. To motivate the model, we use exogenous variation in the exposure of counties to COVID-19 shocks in their social network to show that a 10\% rise in the number of cases and deaths is associated with a 0.15\% and 0.42\% decline in consumption expenditures, respectively. Interestingly, these effects are concentrated	among consumer goods and services that rely more on social contact and are not driven by local shocks. Next, we embed a tractable belief formation mechanism through social communications, a la DeGroot (1974) in an otherwise standard heterogeneous-agent model and calibrate the model to replicate our micro findings. We show a pandemic-augmented version of this model where infections initially hit a fraction of more connected regions and gradually propagated via social network helps generates macroeconomic dynamics more aligned with the empirical patterns of aggregate consumption and cross-sectional heterogeneity. We demonstrate how the dynamic and size of aggregate responses depend on the location of the initial shocks and the structure of the network.
		\end{innerlist}
	}{}
	
	\item  \textit{``Epidemiological Expectations''}, \href{https://github.com/llorracc/EpiExp/blob/master/draft/chapter/Book.pdf}{working paper}, in preparation for \textit{Handbook of Economic Expectations}, with Christopher Carroll, submitted.
	\ifthenelse{\boolean{long}}{
		\begin{innerlist}
			\item[] \textit{Abstract}: `Epidemiological' models of belief formation put social interactions at their core; such models are the main (almost, the only) tool used by non-economists to study the dynamics of beliefs in populations.  We survey the (comparatively) small literature in which economists attempting to model the consequences of beliefs about the future -- `expectations' -- have employed a full-fledged epidemiological approach to explore an economic question.  We draw connections to related work on `contagion,' narrative economics, news/rumor spreading, and the spread of online content. We conclude by arguing that a number of independent developments have converged in ways that may make EE modeling more feasible and more appealing in the past.
		\end{innerlist}
	}{}
	
	\item  \textit{``Perceived (Un)employment Risks Over Business Cycles''}, work in progress, with  William Du and Xincheng Qiu, 2022.
	\ifthenelse{\boolean{long}}{
		\begin{innerlist}
			\item[] \textit{Abstract}: How do income risks such as job loss and finding probabilities as perceived by households change over business cycles? Are the perceptions of income risks reported in the survey consistent with the realizations estimated from the conventional data sources? Answering these questions could help inform both the rapidly rising incomplete-market macro literature that shows the change in income risks is an important driver of business cycle fluctuations and the behavioral extensions of these models. The direct empirical evidence relied upon expectation surveys has been rare, as survey-reported beliefs about income risks are not available until the recent decade. This paper achieves two goals in light of these developments. First, we impute survey-reported beliefs of unemployment risks from the recent decade back to as early as the 1970s, utilizing the cross-survey correlations between the risk beliefs and other macroeconomic expectations available for a longer time span. Second, we document a number of salient facts about the changes in unemployment risk perceptions over business cycles, and also how it compares to realizations estimated from the census data. We found that although the beliefs do reasonably respond to real-time macroeconomic conditions and co-move with their realizations ex-post, households' belief underact to changes of aggregate labor market conditions.
		\end{innerlist}
	}{}
\end{enumerate}   

%%=====================